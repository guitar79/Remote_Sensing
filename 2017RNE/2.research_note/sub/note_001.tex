%\documentclass{research-note-v1.0}
\section*{May, 31, 2017}

\begin{itemize}
	\item{연구 장소 : 과학영재연구센터 627호}
	\item{참여자 : \@Author, \@secondAuthor, \@thirdAuthor, \@fourthAuthor}
\end{itemize}

	\begin{table}[h]
	%	\caption{Research Note.}
		\label{table01}
%		\centering
		\begin{tabular}{|c|c|} 
			\hline
			연구 장소   &   과학영재연구센터 627호 \\ 
			\hline
			참여자   &   {\@firstAuthor}, \@secondAuthor, \@thirdAuthor, \@fourthAuthor  \\
			\hline
		\end{tabular}
	\end{table}

	\subsection*{연구 계획 수립}
							연구 계획을 수립하기 위하여 만남.
							연구 계획을 수립하기 위하여 만남.
								연구 계획을 수립하기 위하여 만남.
									연구 계획을 수립하기 위하여 만남.
									v
	
    \subsection*{역할 분담}
			연구 계획을 수립하기 위하여 만남.
				연구 계획을 수립하기 위하여 만남.
					연구 계획을 수립하기 위하여 만남.
						연구 계획을 수립하기 위하여 만남.
							연구 계획을 수립하기 위하여 만남.
								연구 계획을 수립하기 위하여 만남.
									연구 계획을 수립하기 위하여 만남.
										연구 계획을 수립하기 위하여 만남.
	\subsection*{컴퓨터 셋업}
		연구 계획을 수립하기 위하여 만남.
			연구 계획을 수립하기 위하여 만남.
				연구 계획을 수립하기 위하여 만남.
					연구 계획을 수립하기 위하여 만남.
						연구 계획을 수립하기 위하여 만남.
							연구 계획을 수립하기 위하여 만남.

	\subsection*{Collection Data}
     Data는 다음과 같은 방법으로 order하여 다운로드 받는다.

	\begin{itemize}
		\item{연구 주제의 전반적 관심을 조명.}
		\item{연구 분야의 스페셜 이슈를 조명.}
		\item{해당 이슈를 해결하기 위한 다양한 선행 연구들을 서술.}
		\item{선행 연구들의 한계점을 기술.}
		\item{한계를 극복하기 위한 본 연구의 목적을 밝힘.}
		\item{논문의 구성을 서술 (optional).}
	\end{itemize}


\section*{June, 5, 2017}

\subsection*{선행연구 고찰}

MODIS 관련 연구
\begin{itemize}
	\item{Characteristics of distribution and seasonal variation of aerosol optical depth in eastern China with MODIS products. \cite{ahn2012development}
	\item{Validation of MODIS aerosol optical depth retrieval over land. \cite{chu2002validation}
\end{itemize}

subsection*{분석}

 분석 내용 작성

subsection*{회의}

회의 내용 작성

subsection*{결과}

결과 작성


\subsection*{다음 계획}


\section*{May, 31, 2016}

\subsection*{실험}

실험 내용 작성

\subsection*{분석}

분석 내용 작성

\subsection*{회의}

회의 내용 작성

\subsection*{결과}

결과 작성


\subsection*{다음 계획}
