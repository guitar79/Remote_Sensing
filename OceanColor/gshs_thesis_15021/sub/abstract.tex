\maketitle  % command to print the title page with above variables
\setcounter{page}{1}
%---------------------------------------------------------------------
%                  영문 초록을 입력하시오
%---------------------------------------------------------------------
\begin{abstracts}     %this creates the heading for the abstract page
	\addcontentsline{toc}{section}{Abstract}  %%% TOC에 표시
	\noindent{
		The seasonal variability chlorophyll-a concentration in the Korean East Sea had been obtained by processing SeaWiFS Local Area Coverage(LAC) and Global Area Coverage(GAC) oceancolor data from January 2003 to December 2006. Both data showed similarities in the tendency showing peaks on spring(April) and fall(November). However, the value of chlorophyll-a concentration was different, the maximum/minimum value of LAC and GAC data each being 1.61 mg/m3 / 0.28 mg/m3, and 1.72 mg/m3 / 0.32 mg/m3. Comparing the pixel histogram of LAC and GAC data showed GAC data had more speckle errors. The pixel analysis of chlorophyll-a concentration data on April, 2003 also showed it is more accurate to use LAC data because of its high resolution.
	}
\end{abstracts}

%---------------------------------------------------------------------
%                  국문 초록을 입력하시오
%---------------------------------------------------------------------
\begin{abstractskor}        %this creates the heading for the abstract page
	\addcontentsline{toc}{section}{초록}  %%% TOC에 표시
	\noindent{
	2003년 1월부터 2006년 12월까지 동해의 chlorophyll-a 농도 계절 변동을 SeaWiFS LAC (Local Area Coverage) 및 GAC (Global Area Coverage) 자료를 가공하여 얻었다. 두 자료 모두 유사한 경향을 보여 봄철(4월)과 가을철(11월)에 농도가 최대로 나타났다. 그러나 LAC와 GAC 데이터의 chlorophyll-a 농도 최대값/최소값은 각각 1.61 mg/m3 / 0.28 mg/m3 과 1.72 mg/m3 / 0.32 mg/m3으로 차이가 나타났다. 더 나아가서 LAC와 GAC 데이터의 픽셀 히스토그램을 비교하여 GAC 데이터에서 스펙클 오류가 더 크게 나타남을 확인하였다. 2003년 4월의 엽록소 농도 데이터의 픽셀 분석 역시 해상도가 높은LAC 데이터를 사용하는 것이 더 정확함을 확인하였다. 
	}
\end{abstractskor}


%----------------------------------------------
%   Table of Contents (자동 작성됨)
%----------------------------------------------
\cleardoublepage
\addcontentsline{toc}{section}{Contents}
\setcounter{secnumdepth}{3} % organisational level that receives a numbers
\setcounter{tocdepth}{3}    % print table of contents for level 3
\baselineskip=2.2em
\tableofcontents


%----------------------------------------------
%     List of Figures/Tables (자동 작성됨)
%----------------------------------------------
\cleardoublepage
\clearpage
\listoftables
% 표 목록과 캡션을 출력한다. 만약 논문에 표가 없다면 이 위 줄의 맨 앞에 
% `%' 기호를 넣어서 주석 처리한다.

\cleardoublepage
\clearpage
\listoffigures
% 그림 목록과 캡션을 출력한다. 만약 논문에 그림이 없다면 이 위 줄의 맨 앞에 
% `%' 기호를 넣어서 주석 처리한다.

\cleardoublepage
\clearpage
\listofequations
% 그림 목록과 캡션을 출력한다. 만약 논문에 그림이 없다면 이 위 줄의 맨 앞에 
% `%' 기호를 넣어서 주석 처리한다.


\cleardoublepage
\clearpage
\renewcommand{\thepage}{\arabic{page}}
\setcounter{page}{1}