\section{Introduction}

Algae is an important factor in the marine ecosystem. It adds oxygen to water by photosynthesis, and it also determines the water transparency. However, when nutrients are oversupplied, eutrophication occurs causing overgrowth of algae in water. Such phenomenon is called as algal bloom. It blocks sunlight and also consumes lot of oxygen when it dies and decomposes, causing a negative effect on the marine environment. In addition, measuring the amount of algae in the water is crucial.

Ocean color remote sensing allows us to indirectly measure various matters in the ocean. The Coastal Zone Color Scanner (CZCS) that is the first ocean colour sensor, operated from 1978 to 1986 and the Sea-viewing Wide Field-of-view Sensor (SeaWiFS) has observed global ocean colour distributions for about a decade, from 1998 to 2010, and has provided the scientific community with abundant information for a variety of oceanic application research \cite{kyung2013characteristics, hooker1992An}.

It collects several bands of reflected light from the ocean, which can be calculated to amount of matters using developed algorithms. The data we obtain from such methods is called ocean color data. Ocean color data is important in understanding the temporal and spatial distribution of algae \cite{kimhc2016surface}.

Chlorophyll-a concentration can be calculated using ocean color remote sensing. Since chlorophyll-a is a pigment that is required for photosynthesis, it is also found in algae. Additionally, chlorophyll-a concentration is used as a determinant for the amount of algae in ocean \cite{o2000ocean}. 

There have been several studies about chlorophyll-a concentration in the Korean East Sea, using ocean color remote sensing. Spring bloom and fall bloom were observed, showing a seasonal pattern. The chlorophyll-a concentration was also found to be related with wind, sea surface temperature, and many other variables of the ocean \cite{yamada2004seasonal}. SeaWiFS(Sea-Viewing Wide Field-of-View Sensor) satellite is often used for studying the chlorophyll-a concentration in the Korean East Sea. It was found to have small areas with abnormally high concentration of chlorophyll-a compared to the areas nearby, which is called speckles. There had been studies to correct this error \cite{chae2009characteristics}. 

SeaWiFS creates two types of level 1 data with different spatial resolutions, Local-Area Coverage(LAC) data and Global-Area Coverage(GAC) data. Level-1 data is a data that had gone through radiometric and geometric calibration. LAC is created with full-resolution(1.1km) for local area while GAC is created with low resolution(4.5km) for a global area. They both use the OC4 algorithm and the color index algorithm to be converted to level-2 chlorophyll-a concentration data. Level-2 data is a data of geophysical variables processed from level-1 data. 

(LAC와 GAC 차이를 좀 더 자세히 적고 인용표시)

Although many scientists used SeaWiFS LAC data and GAC data for research, they did not found which is more suitable. There is also no proof that applying same algorithms will give both accurate results. 

The goal of this research are the followings. First, observing the chlorophyll-a concentration variability in the Korean East Sea from 2003 to 2006 using SeaWiFS ocean color data. Second, comparing the chlorophyll-a concentration data of LAC and GAC to find the effect of spatial resolution.

