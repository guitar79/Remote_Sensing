\section{Discussion}
Spring(April) and fall(November) each had 1.2 ~ 1.3 mg/m3, 0.9 ~ 1.0 mg/m3 of chlorophyll-a concentrations which was higher than the other seasons showing similarities with the result of Yamada, K., Ishizaka, J., Yoo, S., Kim, H. C., & Chiba, S. (2004). Analysis of annual variability with the temporal resolution increased to 8-day-mean data showed winter (December, January, February) had higher concentration compared to summer (June, July, August), each with the values of 0.6 mg/m3 and 0.4 mg/m3. This conclude that the Korean East Sea shows clear seasonal differences.

In a perspective of spatial resolution, both the LAC data and the GAC data were suitable for research on the chlorophyll-a concentration variability in the Korean East Sea. However, histogram analysis showed that GAC data had more speckles compared to LAC data. The pixel analysis of chlorophyll-a concentration data on April, 2003 also showed that even with the speckle correction it is more accurate to use LAC data because of its high resolution.

The limit of this research is that speckle errors were not corrected causing the concentration values to be higher than the in-situ data. Moreover, the in-situ data of the area of interest was not measured in this research so it could not directly show whether the LAC data or the GAC data is more accurate. Further research will compare the processed data of LAC and GAC with the in-situ data from previous researches.
