%%%%%%%%%%%%%%%%%%%%%%%%%%%%%%%%%%%%%%%
%%% Research note - Entry
%%% Template by Mikhail Klassen, April 2013
%%% 
%%%%%%%%%%%%%%%%%%%%%%%%%%%%%%%%%%%%%%%

%경기과학고 졸업논문, 알앤이 보고서에서 쓰던 클래스와 적절히 합침.  by 박기현(2017. 6. 5)
\documentclass{research-note-v1.0}

% 경기과학고 로고를 다음과 같이 넣어두세요."./img/logo.png"  by 박기현(2017. 6. 5)

\researchType{창의연구 R\&E  } % 기초 / 심화  /  창의연구 R\&E
%\reporttype{중간} % 중간 / 결과

\researchTitle{MODIS 자료를 이용한 에어로졸의 광학적 두께(AOD) 산출} % 제목 개행 시 \linebreak 사용. \\나 \newline 은 안됨.
\englishTitle{English title}% 제목 개행 시 \linebreak 사용. \\나 \newline 은 안됨.

\author[1] {김병현} % 제 1 저자명
\email[1]{@e-mail.address} % 제 1 저자 이메일
\author[2] {전우치} % 제 2 저자명
\email[2]{@e-mail.address} % 제 2 저자 이메일
\author[3] {아이유} % 제 3 저자명
\email[3]{iu@bogo.sipda} % 제 3 저자 이메일
\author[4] {아이유} % 제 4 저자명
\email[4]{iu@bogo.sipda} % 제 4 저자 이메일
\advisor{박기현} % 지도교사명
\advisorEmail{guitar79@naver.com} % 지도교사 이메일

%\newcommand{\studentName}{연구자 : 김병현 외} %수정하세요.
\newcommand{\institution}{경기과학고등학교} %수정하세요.




\begin{document}
	\title{Research Diary}
		
	\section*{May, 31, 2016}
	\begin{table}[h]
	%	\caption{Research Note.}
	%	\label{table01}
		\centering
			\hspace{3}
		\begin{tabular}{c|c} \hline
			연구 장소   &   과학영재연구센터 627호  \\ \hline
			참여자   &   박기현, /stu01 , /stu02  \\ \hline
		\end{tabular}
	\end{table}

\begin{tabbing}
	\hspace{3}\=\kill
	\> 
\end{tabbing} 

	\subsection*{연구 계획 수립}
							연구 계획을 수립하기 위하여 만남.
							연구 계획을 수립하기 위하여 만남.
								연구 계획을 수립하기 위하여 만남.
									연구 계획을 수립하기 위하여 만남.
									v
	
    \subsection*{역할 분담}
			연구 계획을 수립하기 위하여 만남.
				연구 계획을 수립하기 위하여 만남.
					연구 계획을 수립하기 위하여 만남.
						연구 계획을 수립하기 위하여 만남.
							연구 계획을 수립하기 위하여 만남.
								연구 계획을 수립하기 위하여 만남.
									연구 계획을 수립하기 위하여 만남.
										연구 계획을 수립하기 위하여 만남.
	\subsection*{컴퓨터 셋업}
		연구 계획을 수립하기 위하여 만남.
			연구 계획을 수립하기 위하여 만남.
				연구 계획을 수립하기 위하여 만남.
					연구 계획을 수립하기 위하여 만남.
						연구 계획을 수립하기 위하여 만남.
							연구 계획을 수립하기 위하여 만남.

	\subsection*{Collection Data}
     Data는 다음과 같은 방법으로 order하여 다운로드 받는다.

	\begin{itemize}
		\item{연구 주제의 전반적 관심을 조명.}
		\item{연구 분야의 스페셜 이슈를 조명.}
		\item{해당 이슈를 해결하기 위한 다양한 선행 연구들을 서술.}
		\item{선행 연구들의 한계점을 기술.}
		\item{한계를 극복하기 위한 본 연구의 목적을 밝힘.}
		\item{논문의 구성을 서술 (optional).}
	\end{itemize}


\section*{May, 31, 2016}

\subsection*{실험}

실험 내용 작성

\subsection*{분석}

 분석 내용 작성

\subsection*{회의}

회의 내용 작성

\subsection*{결과}

결과 작성


\subsection*{다음 계획}




\section*{May, 31, 2016}

\subsection*{실험}

실험 내용 작성

\subsection*{분석}

분석 내용 작성

\subsection*{회의}

회의 내용 작성

\subsection*{결과}

결과 작성


\subsection*{다음 계획}



\end{document}